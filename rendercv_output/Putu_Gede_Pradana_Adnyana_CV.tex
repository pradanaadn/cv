\documentclass[10pt, a4paper]{article}

% Packages:
\usepackage[
    ignoreheadfoot, % set margins without considering header and footer
    top=2 cm, % seperation between body and page edge from the top
    bottom=2 cm, % seperation between body and page edge from the bottom
    left=2 cm, % seperation between body and page edge from the left
    right=2 cm, % seperation between body and page edge from the right
    footskip=1.0 cm, % seperation between body and footer
    % showframe % for debugging 
]{geometry} % for adjusting page geometry
\usepackage{titlesec} % for customizing section titles
\usepackage{tabularx} % for making tables with fixed width columns
\usepackage{array} % tabularx requires this
\usepackage[dvipsnames]{xcolor} % for coloring text
\definecolor{primaryColor}{RGB}{9, 16, 87} % define primary color
\usepackage{enumitem} % for customizing lists
\usepackage{fontawesome5} % for using icons
\usepackage{amsmath} % for math
\usepackage[
    pdftitle={Putu Gede Pradana Adnyana's CV},
    pdfauthor={Putu Gede Pradana Adnyana},
    pdfcreator={LaTeX with RenderCV},
    colorlinks=true,
    urlcolor=primaryColor
]{hyperref} % for links, metadata and bookmarks
\usepackage[pscoord]{eso-pic} % for floating text on the page
\usepackage{calc} % for calculating lengths
\usepackage{bookmark} % for bookmarks
\usepackage{lastpage} % for getting the total number of pages
\usepackage{changepage} % for one column entries (adjustwidth environment)
\usepackage{paracol} % for two and three column entries
\usepackage{ifthen} % for conditional statements
\usepackage{needspace} % for avoiding page brake right after the section title
\usepackage{iftex} % check if engine is pdflatex, xetex or luatex

% Ensure that generate pdf is machine readable/ATS parsable:
\ifPDFTeX
    \input{glyphtounicode}
    \pdfgentounicode=1
    \usepackage[T1]{fontenc}
    \usepackage[utf8]{inputenc}
    \usepackage{lmodern}
\fi

\usepackage{charter}

% Some settings:
\AtBeginEnvironment{adjustwidth}{\partopsep0pt} % remove space before adjustwidth environment
\pagestyle{empty} % no header or footer
\setcounter{secnumdepth}{0} % no section numbering
\setlength{\parindent}{0pt} % no indentation
\setlength{\topskip}{0pt} % no top skip
\setlength{\columnsep}{0.15cm} % set column seperation
\makeatletter
\let\ps@customFooterStyle\ps@plain % Copy the plain style to customFooterStyle
\patchcmd{\ps@customFooterStyle}{\thepage}{
    \color{gray}\textit{\small Putu Gede Pradana Adnyana - Page \thepage{} of \pageref*{LastPage}}
}{}{} % replace number by desired string
\makeatother
\pagestyle{customFooterStyle}

\titleformat{\section}{\needspace{4\baselineskip}\bfseries\large}{}{0pt}{}[\vspace{1pt}\titlerule]

\titlespacing{\section}{
    % left space:
    -1pt
}{
    % top space:
    0.3 cm
}{
    % bottom space:
    0.2 cm
} % section title spacing

\renewcommand\labelitemi{$\vcenter{\hbox{\small$\bullet$}}$} % custom bullet points
\newenvironment{highlights}{
    \begin{itemize}[
        topsep=0.10 cm,
        parsep=0.10 cm,
        partopsep=0pt,
        itemsep=0pt,
        leftmargin=0 cm + 10pt
    ]
}{
    \end{itemize}
} % new environment for highlights


\newenvironment{highlightsforbulletentries}{
    \begin{itemize}[
        topsep=0.10 cm,
        parsep=0.10 cm,
        partopsep=0pt,
        itemsep=0pt,
        leftmargin=10pt
    ]
}{
    \end{itemize}
} % new environment for highlights for bullet entries

\newenvironment{onecolentry}{
    \begin{adjustwidth}{
        0 cm + 0.00001 cm
    }{
        0 cm + 0.00001 cm
    }
}{
    \end{adjustwidth}
} % new environment for one column entries

\newenvironment{twocolentry}[2][]{
    \onecolentry
    \def\secondColumn{#2}
    \setcolumnwidth{\fill, 4.5 cm}
    \begin{paracol}{2}
}{
    \switchcolumn \raggedleft \secondColumn
    \end{paracol}
    \endonecolentry
} % new environment for two column entries

\newenvironment{threecolentry}[3][]{
    \onecolentry
    \def\thirdColumn{#3}
    \setcolumnwidth{, \fill, 4.5 cm}
    \begin{paracol}{3}
    {\raggedright #2} \switchcolumn
}{
    \switchcolumn \raggedleft \thirdColumn
    \end{paracol}
    \endonecolentry
} % new environment for three column entries

\newenvironment{header}{
    \setlength{\topsep}{0pt}\par\kern\topsep\centering\linespread{1.5}
}{
    \par\kern\topsep
} % new environment for the header

\newcommand{\placelastupdatedtext}{% \placetextbox{<horizontal pos>}{<vertical pos>}{<stuff>}
  \AddToShipoutPictureFG*{% Add <stuff> to current page foreground
    \put(
        \LenToUnit{\paperwidth-2 cm-0 cm+0.05cm},
        \LenToUnit{\paperheight-1.0 cm}
    ){\vtop{{\null}\makebox[0pt][c]{
        \small\color{gray}\textit{Last updated in December 2024 (AI/ML Engineer)}\hspace{\widthof{Last updated in December 2024 (AI/ML Engineer)}}
    }}}%
  }%
}%

% save the original href command in a new command:
\let\hrefWithoutArrow\href

% new command for external links:
\renewcommand{\href}[2]{\hrefWithoutArrow{#1}{\ifthenelse{\equal{#2}{}}{ }{#2 }\raisebox{.15ex}{\footnotesize \faExternalLink*}}}


\begin{document}
    \newcommand{\AND}{\unskip
        \cleaders\copy\ANDbox\hskip\wd\ANDbox
        \ignorespaces
    }
    \newsavebox\ANDbox
    \sbox\ANDbox{$|$}

    \placelastupdatedtext
    \begin{header}
        \fontsize{15pt}{15pt}\selectfont Putu Gede Pradana Adnyana

        \vspace{5 pt}

        \normalsize
        \mbox{{\footnotesize\faMapMarker*}\hspace*{0.13cm}Badung, Bali, Indonesia}%
        \kern 5.0 pt%
        \AND%
        \kern 5.0 pt%
        \mbox{\hrefWithoutArrow{mailto:work.pradanaadn@gmail.com}{{\footnotesize\faEnvelope[regular]}\hspace*{0.13cm}work.pradanaadn@gmail.com}}%
        \kern 5.0 pt%
        \AND%
        \kern 5.0 pt%
        \mbox{\hrefWithoutArrow{tel:+62-812-3737-4363}{{\footnotesize\faPhone*}\hspace*{0.13cm}0812-3737-4363}}%
        \kern 5.0 pt%
        \AND%
        \kern 5.0 pt%
        \mbox{\hrefWithoutArrow{https://pradanaadn.github.io/about/}{{\footnotesize\faLink}\hspace*{0.13cm}pradanaadn.github.io/about}}%
        \kern 5.0 pt%
        \AND%
        \kern 5.0 pt%
        \mbox{\hrefWithoutArrow{https://linkedin.com/in/pradanaadn}{{\footnotesize\faLinkedinIn}\hspace*{0.13cm}pradanaadn}}%
        \kern 5.0 pt%
        \AND%
        \kern 5.0 pt%
        \mbox{\hrefWithoutArrow{https://github.com/pradanaadn}{{\footnotesize\faGithub}\hspace*{0.13cm}pradanaadn}}%
    \end{header}

    \vspace{5 pt - 0.3 cm}


    \section{Summary}



        
        \begin{onecolentry}
            Innovative Computer Engineering graduate with 1+ year of experience in AI-powered solutions and scalable software development. Proficient in Python with 4 years of experience and a strong background in deploying machine learning models, optimizing systems, and implementing cutting-edge technologies. Successfully contributed as an Artificial Intelligence Engineer Intern at Ruangguru, enabling seamless LLM access for all stakeholders. Willing to relocate for opportunities that align with my skills and career goa
        \end{onecolentry}


    
    \section{Education}



        
        \begin{twocolentry}{
            Sept 2020 – Aug 2024
        }
            \textbf{Udayana University}, S.T in Electrical and Computer Engineering\end{twocolentry}

        \vspace{0.10 cm}
        \begin{onecolentry}
            \begin{highlights}
                \item \textbf{GPA}: 3.97/4.0 (\href{https://drive.google.com/file/d/1pP8v7Sbi2i_VDCOLNyjXUbzSdkNwGF2a/view?usp=sharing}{Transcripts})
                \item \textbf{Best Graduates} of the Faculty of Engineering, Udayana University at the 163rd Graduation
                \item \textbf{Coursework:} Discrete Mathematics, Calculus, Data Structure, Computer Architecture, Software Engineering, Machine Learning, Digital Image Processing, Big Data, OOP, Database, Computer Network, Information Technology Project Management
            \end{highlights}
        \end{onecolentry}


        \vspace{0.2 cm}

        \begin{twocolentry}{
            Mar 2024 – July 2024
        }
            \textbf{Mastering AI: From Foundations to Applications by Ruangguru}, Machine Learning Engineering\end{twocolentry}

        \vspace{0.10 cm}
        \begin{onecolentry}
            \begin{highlights}
                \item GPA: 90/100 (\href{https://drive.google.com/file/d/1zavbSHSpPOePGauqHs-WYxpkUhOqOrRa/view}{Transcripts})
                \item \textbf{Coursework:}  Python, Data Cleaning, Data Visualization, Statistic, Linear algebra and calculus, Machine Learning, Deep Learning, CNN and Computer Vision, Pytorch, NLP and Transformer, MLOps
            \end{highlights}
        \end{onecolentry}


        \vspace{0.2 cm}

        \begin{twocolentry}{
            Feb 2023 – July 2023
        }
            \textbf{Bangkit by Google, Goto and Traveloka}, Machine Learning\end{twocolentry}

        \vspace{0.10 cm}
        \begin{onecolentry}
            \begin{highlights}
                \item GPA: 95/100 (\href{https://drive.google.com/file/d/1JOm2b6ws9PepLZTCV8uZDdDENN64KT7d/view?usp=sharing}{Transcripts})
                \item Distinction Graduates, 10\% of more than 5,000.
                \item \textbf{Coursework:} Python, Data analytics, Mathematics for Machine Learning (Linear Algebra, Calculus), Machine Learning, Deep learning with Tensorflow, ML Deployment
            \end{highlights}
        \end{onecolentry}



    
    \section{Technical Skills}



        
        \begin{onecolentry}
            \textbf{Language:} Python (Advanced), C++ (Intermediate), Javascript (Intermediate), SQL (Intermediate), Bash (Intermediate)
        \end{onecolentry}

        \vspace{0.2 cm}

        \begin{onecolentry}
            \textbf{Data Analysis and Visualization:} Pandas, Numpy, Seaborn, Matplotlib, Plotly, Tableau
        \end{onecolentry}

        \vspace{0.2 cm}

        \begin{onecolentry}
            \textbf{Machine Learning Modeling:} Pytorch, Tensorflow, Sklearn, Huggingface Transformer
        \end{onecolentry}

        \vspace{0.2 cm}

        \begin{onecolentry}
            \textbf{Large Language Model:} Langchain, VertexAI, OpenAI, Anthropic
        \end{onecolentry}

        \vspace{0.2 cm}

        \begin{onecolentry}
            \textbf{API and Web Development Framework:} Flask, FastAPI, Django REST API, Laravel
        \end{onecolentry}

        \vspace{0.2 cm}

        \begin{onecolentry}
            \textbf{Deployment:} Docker, Git, Gthub Action, Streamlit, Gradio
        \end{onecolentry}

        \vspace{0.2 cm}

        \begin{onecolentry}
            \textbf{Cloud Service:} Azure AI Service, Google Cloud Platform
        \end{onecolentry}


    
    \section{Experience}



        
        \begin{twocolentry}{
            Sept 2024 – Dec 2024
        }
            \textbf{Artificial Intelligence Engineer Intern}, Ruang Guru -- Jakarta, Indonesia\end{twocolentry}

        \vspace{0.10 cm}
        \begin{onecolentry}
            \begin{highlights}
                \item Led the development of a centralized system integrating AI models such as OpenAI, Gemini, and Anthropic, streamlining access for product and development teams.
                \item Developed REST API endpoints for a Coding Assistant, improving engineering team access.
                \item Implemented CI/CD pipelines to automate the deployment process, reducing deployment errors by 25\%.
                \item Optimized system performance by enhancing architecture, resulting in a 30\% increase in efficiency.
                \item Collaborated with cross-functional teams to ensure the scalability and reliability of IT solutions in dynamic environments.
                \item Showcased expertise in system optimization, driving improvements in reliability to support Ruang Guru's digital transformation initiatives.
            \end{highlights}
        \end{onecolentry}


        \vspace{0.2 cm}

        \begin{twocolentry}{
            Aug 2024 – Aug 2024
        }
            \textbf{Product Development Intern}, XL Axiata (X-Camp) -- Jakarta, Indonesia\end{twocolentry}

        \vspace{0.10 cm}
        \begin{onecolentry}
            \begin{highlights}
                \item Contributed to the integration of RTSP cameras for object detection and optimized multithreaded data processing using CUDA, improving system performance by 40\% and enabling real-time analytics.
                \item Led research on deploying YOLO models on Jetson Nano for object detection, enhancing speed and accuracy in AI-driven solutions.
                \item Configured MQTT protocols and integrated them with ThingsBoard, ensuring seamless data communication and system reliability.
                \item Improved scalability of monitoring systems through optimized architecture and AI deployment strategies.
                \item Delivered a 30\% improvement in task execution speed, aligning with XL Axiata's goals for digital transformation and operational efficiency.
            \end{highlights}
        \end{onecolentry}



    
    \section{Certification}



        
        \begin{onecolentry}
            \href{https://www.credential.net/e5709acf-2219-4c59-8aaf-987215d069a1}{TensorFlow Developer Certificate}, \href{https://learn.microsoft.com/api/credentials/share/id-id/PradanaAdnyana-5811/2A28EA83F4251FC4?sharingId=D4E348B6E54ABB69}{Microsoft Certified: Azure AI Engineer Associate}
        \end{onecolentry}


    
    \section{Soft Skills}



        
        \begin{onecolentry}
            Project Management, Communication, Agile development, Time management, Problem Solving, Data Analytics, Leadership, Continuous learning, Cross-Functional Collaboration, Data-driven Mindset
        \end{onecolentry}


    
    \section{Language Proficiency}



        
        \begin{onecolentry}
            Indonesia (Native), English (Proficient, TOEFL ITP Score >500) [\href{https://drive.google.com/file/d/1w1m4vxTjOT2xaDfYtqPXp7TDrWySaDbE/view?usp=sharing}{Certificate}]
        \end{onecolentry}


    
    \section{Projects}



        
        \begin{twocolentry}{
            (\href{https://pitch.com/v/mangorenai-x6n96f}{Project Presentation})
        }
            \textbf{Trash Object Detection - AI-Based Waste Audit and Assistant}\end{twocolentry}

        \vspace{0.10 cm}
        \begin{onecolentry}
            \begin{highlights}
                \item Tackled waste management issues in Bandung, addressing landfill sites exceeding 800\% capacity, with AI-powered detection and monitoring solutions.
                \item Developed Catch The Trash, achieving 92\% precision using YOLOv8 to classify waste types and provide actionable reuse recommendations.
                \item Created MamangHijau, an interactive chatbot powered by Qwen-MAX, to educate users on waste handling, regulations, and utilization.
                \item Designed a real-time dashboard to monitor waste quantities and types, enabling data-driven decisions for government and enterprise users.
                \item Provided actionable insights, such as identifying 60\% organic waste, leading to better infrastructure planning like composting facilities.
                \item Tools Used: Python, Streamlit, Alibaba Cloud, Ultralytics, OpenCV
            \end{highlights}
        \end{onecolentry}


        \vspace{0.2 cm}

        \begin{twocolentry}{
            (\href{https://github.com/pradanaadn/covid-19-detection}{Project Repository})
        }
            \textbf{COVID-19 Detection via Chest X-rays}\end{twocolentry}

        \vspace{0.10 cm}
        \begin{onecolentry}
            \begin{highlights}
                \item Tackled the challenge of automating COVID-19 detection using chest X-rays with a deep learning model.
                \item Led the development and training of a CNN model using PyTorch and a fine-tuned ResNet-50 architecture.
                \item Preprocessed images (224x224 size, normalized, augmented) and implemented training with Adam optimizer and cross-entropy loss.
                \item Achieved 92\% validation accuracy after 10 epochs by fine-tuning pretrained layers.
                \item Built a scalable pipeline for real-time predictions with confidence scores on unseen data.
                \item Tools Used: Python, Pytorch, Transformer, Gradio
            \end{highlights}
        \end{onecolentry}


        \vspace{0.2 cm}

        \begin{twocolentry}{
            (\href{https://github.com/Talenta-AI-2}{Project Repository})
        }
            \textbf{UmMeals – Maternal and Child Nutrition App}\end{twocolentry}

        \vspace{0.10 cm}
        \begin{onecolentry}
            \begin{highlights}
                \item Developed an app supporting nutrition tracking for pregnant women and toddlers, featuring personalized monitoring and early stunting detection to provide better health insights.
                \item Managed all project phases from ideation and research to design and deployment, resulting in selection as a top 6 finalist out of 15 teams in the Skilvul, Biji-biji Initiative, and Microsoft Innovation Challenge 2024.
                \item Utilized Python, SQL with ORM using SQLAlchemy, Streamlit, Microsoft Azure AI, and Power BI to enhance app functionalities.
                \item Achieved recognition for innovative solutions and impactful design in the nutrition tracking domain, contributing to improved health outcomes.
                \item Tools Used: Python, SQL with ORM using SQLAlchemy, Streamlit, Microsoft Azure AI, Power BI 
            \end{highlights}
        \end{onecolentry}



    

\end{document}